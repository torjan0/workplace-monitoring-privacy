\documentclass[11pt]{article}

\title{Research Topic Abstract: Workplace Monitoring, Privacy, and Power}
\author{Maksim Hayder}
\date{January 10, 2026}

\begin{document}
\maketitle

\begin{abstract}
Employee monitoring software has expanded from time clocks to pervasive
surveillance systems that log keystrokes, capture screenshots, track
location, and generate productivity scores. This project will examine how
these tools reshape worker privacy, autonomy, and informed consent in
contemporary workplaces. The central research question is: under what
conditions does monitoring become ethically abusive, especially when power
imbalances limit meaningful consent, and what safeguards can align
organizational security needs with employee rights?

The study will map common monitoring capabilities and the rationales
employers cite (security, compliance, performance management), then analyze
their privacy impacts using frameworks from information ethics, labor
studies, and consent theory. Particular attention will be paid to contexts
where workers have limited alternatives or where monitoring is bundled into
essential job functions, creating coerced or uninformed consent. The
research will evaluate thresholds for abuse, including excessive scope,
continuous collection, opaque scoring, secondary use of data, and lack of
appeal or redress.

The project will propose practical policy and technical safeguards such as
data minimization, clear notice and explanation of monitoring logic,
collective bargaining inputs, meaningful opt-out mechanisms, purpose
limitation, and auditable access controls. The goal is a set of criteria
that helps organizations protect legitimate interests while preserving
worker dignity, autonomy, and trust.
\end{abstract}

\end{document}
